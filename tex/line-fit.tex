\documentclass{article}
\usepackage{amssymb,amsmath,blindtext,listings,graphicx}
\title{Linear Least Squares Line Fitting Calculus}
\date{2025 March}
\author{Sean Brennan \\ www.zettix.com \\ github.com/zettix/LeastSquares}
\begin{document}
\maketitle


\section{Introduction}
This is a description of fitting a line to a collection of
two dimensional points $(x,y)$ by minimizing the error defined
as the square of the differences between the line $y = mx + b$ and each $y_i$ for every $x_i$ in the point set, when added together.  Below is an example of
a solution of the best fitting line given the set of points below.

\includegraphics[width=110mm]{whiteplot.png}

To find the line, we first define our error function. Described below,
it is sometimes called the sum of residuals. The important idea is that
it is a function, and thus has parameters and a derivative.

To find the minimum, we take the derivative, or gradient, and set it to zero.
This is possible because the error function is similar to the function
$z = (x - y)^2$, which is convex and increases as $x$ and $y$ differ,
thus the minimum value has gradient zero.

\section{The error function}

The error function is defined as the differences between
the line's $y$ values and the $y$ values of the data set squared for each pair.
\begin{equation}
\sum_{i=1}^{n}\left(m x_i + b - y_i\right)^2
\end{equation}

We can see this as a function $f$ taking two variables, $m$ and $b$:
\begin{equation}
f(m, b) = \sum_{i=1}^{n}\left(m x_i + b - y_i\right)^2
\end{equation}


\section{Finding the minimum: Simplify}

The current definition of $f(m, b)$ would require the chain rule for a
derivative, in a sum, things we can avoid by trying to isolate our terms
by carrying out the square, isolating the sums from our parameters ${\bf m}$
and ${\bf b}$ and separating constants from our variables.

Dropping the sum for the moment, we expand:
\begin{equation}
\begin{split}
f(m, b) & = \left(m x_i + b - y_i\right)^2\\
 & = \left(m x_i + b - y_i\right)\left(m x_i + b - y_i\right)\\
 & = m x_i \left(m x_i + b - y_i\right) + b \left(m x_i + b - y_i\right) - y_i \left(m x_i + b - y_i\right) \\
 & = m x_i m x_i + m x_i b - m x_i y_i + b m x_i + b^2 - b y_i - y_i m x_i - y_i b + y_i^2 \\
 & = m^2 x_i^2 + m b x_i  - m x_i y_i + m b x_i + b^2 - b y_i - m y_i x_i - b y_i + y_i^2 \\
 & = m^2 x_i^2 + 2 m b x_i  - 2 m x_i y_i - 2 b y_i + b^2 + y_i^2 
\end{split}
\end{equation}
If we add the sum back we get:
\begin{equation}
f(m, b) = \sum_{i=1}^n \left( m^2 x_i^2 + 2 m b x_i  - 2 m x_i y_i - 2 b y_i + b^2 + y_i^2 \right)
\end{equation}
Expanding the sums, carefully:
\begin{equation}
f(m, b) = \sum_{i=1}^n m^2 x_i^2 + \sum_{i=1}^n 2 m b x_i  - \sum_{i=1}^n 2 m x_i y_i - \sum_{i=1}^n 2 b y_i + \sum_{i=1}^n b^2 + \sum_{i=1}^n y_i^2 
\end{equation}
Then factoring and simplifying:
\begin{equation}
\label{biggun}
f(m, b) = m^2 \sum_{i=1}^n x_i^2 + 2 m b \sum_{i=1}^n  x_i  - 2 m \sum_{i=1}^n x_i y_i - 2 b \sum_{i=1}^n y_i + n b^2 + \sum_{i=1}^n y_i^2 
\end{equation}

\section{Summed Constants}

We know the sums using $x_i$ and $y_i$ are constant, that is, we cannot change them, we are simply using them to define our gradient.
So let's rename them for our calculations, to make them much easier to work with.
\begin{equation}
\chi = \sum_{i=1}^n x_i
\end{equation}
And
\begin{equation}
\Omega = \sum_{i=1}^n x_i^2
\end{equation}
With
\begin{equation}
\upsilon = \sum_{i=1}^n y_i
\end{equation}
And finally
\begin{equation}
\phi = \sum_{i=1}^n x_i y_i
\end{equation}

Equation \eqref{biggun} becomes much shorter:
\begin{equation}
\label{shorty}
f(m, b) = m^2 \Omega + 2 m b \chi - 2 m \phi - 2 b \nu + n b^2 + \sum_{i=1}^n y_i^2 
\end{equation}

The final sum $\Sigma y_i^2$ does not involve our parameters $m$ or $b$ so we don't simplify it, we will ignore it.

\section{Finding the minimum: derivatives of  $f(m, b)$}
Now let's get the gradient by taking partial derivatives of \eqref{shorty} using our two knobs: $m$ and $b$:
Let us first solve for $m$.
\begin{equation}
\frac{\partial f(m,b)}{\partial m} = 2 m \Omega + 2 b \chi  - 2 \phi
\end{equation}
Then, by $b$
\begin{equation}
\frac{\partial f(m,b)}{\partial b} = 2 m \chi - 2 \nu + 2 n b
\end{equation}
Setting both to zero, and dividing by two, we get:
\begin{equation}
\label{dm}
\frac{\partial f(m,b)}{\partial m} = m \Omega + b \chi - \phi = 0
\end{equation}
And
\begin{equation}
\label{db}
\frac{\partial f(m,b)}{\partial b} =  m \chi + nb - \nu = 0
\end{equation}
Find a common factor for ${\bf b}$ by multiplying \eqref{dm} by ${\bf n}$ and \eqref{db} by ${\bf -\chi}$
\begin{equation}
m {\bf n}  \Omega +  b \chi {\bf n}  -  \phi {\bf n} = 0
\end{equation}
And
\begin{equation}
- m {\bf \chi^2}  - b {\bf \chi} n  +  \upsilon {\bf \chi} = 0
\end{equation}
Adding them and removing the $-b\chi n$ term.
\begin{equation}
{\bf m} \left( n \Omega - \chi^2 \right) + \upsilon \chi - \phi n = 0
\end{equation}
Move $\nu \chi$ and $\phi n$ to the other side, and divide by
$n \Omega - \chi^2$ and we get the first half of our solution, ${m}$
\begin{equation}
{\bf m} = \frac{\phi n - \upsilon \chi }{n \Omega - \chi^2}
\end{equation}

Should we divide by zero, the slope will become $\infty$!
We deal with that in the next section.
Should $n \Omega - \chi^2$ not be zero, we can find $b$ with:
\begin{equation}
{\bf b} \chi = \phi - m \Omega 
\end{equation}
Finally:
\begin{equation}
{\bf b} = \frac{\phi - m \Omega}{\chi}
\end{equation}

\section{Corner cases}

Notice the two pitfalls:  $\chi = 0$ and $n \Omega - \chi^2 = 0$!   What
do these mean?   $\chi = 0$ implies:
\begin{equation}
\sum_{i=1}^n x_i = 0
\end{equation}
This is not difficult.  Consider the points: $(-1, 1), (0, 2), (1, 3)$.
Starting at $(-1, 1)$, the line has slope 1 and $y$ intercept of 2.
Nevertheless, the sum of ${\bf x_i}$ is zero i.e. ${\chi}$ is zero and the
equation for ${\bf b}$ fails.

Hence we try an alternate strategy: solve for ${\bf b}$ instead of ${\bf m}$.
Find a common factor for ${\bf m}$ by multiplying \eqref{dm} by ${\bf \chi}$
 and \eqref{db} by ${\bf -\Omega}$

\begin{equation}
 m \Omega \chi + b \chi^2 - \phi \chi = 0
\end{equation}

\begin{equation}
- m \chi \Omega - n b \Omega + \nu \Omega = 0
\end{equation}

Add them up, and we get:
\begin{equation}
 b \chi^2 - n b \Omega + \nu \Omega - \phi \chi  = 0
\end{equation}

Collecting ${\bf b}$ and moving constants to the other side:
\begin{equation}
b (\chi^2 - n \Omega) =  \phi \chi - \nu \Omega
\end{equation}

We are in a contingency, where $\chi = 0$, so let us exploit this to simplify.
\begin{equation}
n \Omega b =  \nu \Omega
\end{equation}

Immediately:
\begin{equation}
b = \frac{\nu \Omega}{n \Omega} = \frac{\nu}{n}
\end{equation}

Things get even more weird using \eqref{db} with $\chi = 0$ to find ${\bf m}$:
\begin{equation}
m \Omega - \phi = 0
\end{equation}

And hence:
\begin{equation}
m = \frac{\phi}{\Omega}
\end{equation}

Now to deal with pitfall two: $n \Omega - \chi^2 = 0$, or:
\begin{equation}
n  \left( \sum_{i=1}^n x_i^2 \right) - \left( \sum_{i=1}^n x_i \right)^2 = 0
\end{equation}

For example, a single point will create that problem, and a single point
has no solution, nor does a set where all the $x$ values are identical. Here is what we want to avoid:
\begin{equation}
n  \left( \sum_{i=1}^n x_i^2 \right) = \left( \sum_{i=1}^n x_i \right)^2
\end{equation}

If $\mid{\bf \vec{x} }\mid > 0$ then we cannot have the two sums equal due to
the  Cauchy–Schwarz Inequality:

\begin{equation}
\mid\left<{\bf u},{\bf v}\right>\mid^2 \
\le \left<{\bf u},{\bf u}\right>\cdot\left<{\bf v},{\bf v}\right>
\end{equation} 
where ${\bf u}$ is $\vec{x}$ and ${\bf v}$ is $[1, 1, 1, ..., 1]$, and
$< , >$ is the inner (dot) product, while the actual dot($\cdot$) is a
scalar multiply.   Hence we see that $<{\bf v}, {\bf v}> = n$, and the others
match the above equation as well.

So if we get an equality, we simply report an error.

\vfill

\section{Code}
A function to obtain ${\bf m}$ and ${\bf b}$ is given below, as you can see,
it is quite short.
\lstset{language=Python}
\begin{lstlisting}[frame=single]
def FindLine(points=[]):
    """points is a 2-D array
        [[x, y], [x, y], ...]"""
    chi = 0.0
    omega = 0.0
    nu = 0.0
    phi = 0.0

    n = len(points)
    for x, y in points:
      chi += x
      omega += x * x
      nu += y
      phi += x * y

    if chi == 0.0:  # special case.
      m = phi / omega
      b = nu / n
      return (m, b)

    d = n * omega - chi * chi
    if d == 0.0:
      raise Exception("Bad data!")
    m = (phi * n - nu * chi) / d
    b = (phi - m * omega) / chi
    return (m, b)
\end{lstlisting}

\section{Conclusion}

By using partial derivatives and some algebra, it is fairly easy to
understand how to fit a line to a collection of points.  This is
called Gradient Descent and is used in Inverse Kinematics and
Artificial Intelligence.

\end{document}
